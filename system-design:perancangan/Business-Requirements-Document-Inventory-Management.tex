\documentclass[a4paper,12pt]{article}
\usepackage{hyperref}               
\usepackage{enumitem}               
\usepackage{verbatim}               
\usepackage{geometry}               
\geometry{margin=1in}
\usepackage{titlesec}               

\title{Business Requirements Document (BRD) - Solo Inventory Manager}
\author{}
\date{}

\begin{document}

\maketitle
\contentsline {section}{\numberline {1}Project Overview}{2}{section.1}%
\contentsline {section}{\numberline {2}Tujuan Proyek}{2}{section.2}%
\contentsline {section}{\numberline {3}Fitur Utama (Mobile App - Flutter)}{2}{section.3}%
\contentsline {section}{\numberline {4}Persyaratan Fungsional}{3}{section.4}%
\contentsline {subsection}{\numberline {4.1}Autentikasi}{3}{subsection.4.1}%
\contentsline {subsection}{\numberline {4.2}Manajemen Produk}{3}{subsection.4.2}%
\contentsline {subsection}{\numberline {4.3}Scan Barcode}{3}{subsection.4.3}%
\contentsline {subsection}{\numberline {4.4}Notifikasi}{3}{subsection.4.4}%
\contentsline {subsection}{\numberline {4.5}Laporan}{3}{subsection.4.5}%
\contentsline {section}{\numberline {5}Persyaratan Non-Fungsional}{4}{section.5}%
\contentsline {section}{\numberline {6}Stack Teknologi}{4}{section.6}%
\contentsline {subsection}{\numberline {6.1}Keuntungan Penggunaan GetX}{4}{subsection.6.1}%
\contentsline {section}{\numberline {7}Arsitektur Aplikasi}{4}{section.7}%
\contentsline {section}{\numberline {8}Integrasi dengan Supabase}{5}{section.8}%
\contentsline {subsection}{\numberline {8.1}HTTP Request dengan Supabase}{5}{subsection.8.1}%
\contentsline {section}{\numberline {9}Desain Database (Supabase)}{6}{section.9}%
\contentsline {subsection}{\numberline {9.1}Tabel Users}{6}{subsection.9.1}%
\contentsline {subsection}{\numberline {9.2}Tabel Products}{6}{subsection.9.2}%
\contentsline {subsection}{\numberline {9.3}Tabel Transactions}{6}{subsection.9.3}%
\contentsline {section}{\numberline {10}Manajemen Risiko}{7}{section.10}%
\newpage

\section{Project Overview}
\begin{itemize}[leftmargin=1.5cm]
    \item \textbf{Nama Proyek:} Solo Inventory Manager
    \item \textbf{Platform:} Mobile (Flutter 3.27) dengan fokus pada iOS
    \item \textbf{Target Pengguna:} UMKM dengan 10-100 produk
    \item \textbf{Tim Pengembang:} Gheraldy Moses Tarigan
    \item \textbf{Periode Pengembangan:} 1 Minggu
\end{itemize}

\section{Tujuan Proyek}
\begin{itemize}[leftmargin=1.5cm]
    \item Memantau stok barang secara real-time
    \item Mencatat transaksi masuk/keluar barang
    \item Memberi notifikasi otomatis saat stok minimum
    \item Membuat laporan bulanan sederhana
\end{itemize}

\section{Fitur Utama (Mobile App - Flutter)}
\begin{enumerate}[leftmargin=1.5cm]
    \item \textbf{Autentikasi}
    \begin{itemize}
        \item Login dengan username/password
        \item Register akun baru
        \item Reset password
    \end{itemize}
    \item \textbf{Manajemen Produk}
    \begin{itemize}
        \item Tambah, edit, lihat, dan hapus produk
        \item Filter produk berdasarkan kategori
    \end{itemize}
    \item \textbf{Scan Barcode}
    \begin{itemize}
        \item Scan barcode untuk identifikasi produk cepat
        \item Input stok via kamera
    \end{itemize}
    \item \textbf{Notifikasi}
    \begin{itemize}
        \item Peringatan stok rendah
    \end{itemize}
    \item \textbf{Laporan}
    \begin{itemize}
        \item Laporan bulanan sederhana
        \item Statistik inventaris dasar
    \end{itemize}
\end{enumerate}

\section{Persyaratan Fungsional}
\subsection{Autentikasi}
\begin{itemize}[leftmargin=1.5cm]
    \item Sistem harus memungkinkan pengguna membuat akun baru dengan username dan password.
    \item Sistem harus memungkinkan pengguna login dengan kredensial yang sudah terdaftar.
    \item Sistem harus memungkinkan pengguna reset password jika lupa.
    \item Sistem harus menjaga sesi login pengguna.
\end{itemize}

\subsection{Manajemen Produk}
\begin{itemize}[leftmargin=1.5cm]
    \item Sistem harus memungkinkan pengguna menambah produk baru dengan nama, kode, harga, stok, dan kategori.
    \item Sistem harus memungkinkan pengguna mengedit produk yang ada.
    \item Sistem harus memungkinkan pengguna menghapus produk.
    \item Sistem harus menampilkan daftar produk dengan opsi filter dan pencarian.
\end{itemize}

\subsection{Scan Barcode}
\begin{itemize}[leftmargin=1.5cm]
    \item Sistem harus memungkinkan pengguna melakukan scan barcode produk.
    \item Sistem harus dapat mencari produk berdasarkan hasil scan barcode.
    \item Sistem harus memungkinkan penambahan stok dengan memindai barcode.
\end{itemize}

\subsection{Notifikasi}
\begin{itemize}[leftmargin=1.5cm]
    \item Sistem harus mengirim notifikasi saat stok produk mencapai batas minimum.
    \item Sistem harus menampilkan daftar produk dengan stok rendah.
\end{itemize}

\subsection{Laporan}
\begin{itemize}[leftmargin=1.5cm]
    \item Sistem harus menghasilkan laporan bulanan tentang pergerakan inventaris.
    \item Sistem harus menampilkan statistik dasar seperti produk terlaris dan produk dengan perputaran lambat.
\end{itemize}

\section{Persyaratan Non-Fungsional}
\begin{itemize}[leftmargin=1.5cm]
    \item Optimasi untuk iOS dan macOS development.
    \item Waktu respons aplikasi $<$ 1 detik.
    \item Keamanan: Enkripsi data pengguna.
    \item Koneksi internet dibutuhkan karena full menggunakan Supabase.
\end{itemize}

\section{Stack Teknologi}
\begin{itemize}[leftmargin=1.5cm]
    \item \textbf{Frontend:} Flutter dengan komponen UI library (misalnya, Flutter UI Kit, GetX UI).
    \item \textbf{State Management:} GetX.
    \item \textbf{Database:} Supabase (cloud).
    \item \textbf{API:} GetX HTTP client atau Dio untuk HTTP requests ke Supabase.
    \item \textbf{Autentikasi:} Supabase Auth.
    \item \textbf{Lainnya:} Camera plugin, Barcode Scanner.
\end{itemize}

\subsection{Keuntungan Penggunaan GetX}
GetX sangat berguna untuk proyek ini karena:
\begin{itemize}[leftmargin=1.5cm]
    \item State management yang lebih mudah dipelajari dibandingkan BLoC.
    \item Menyediakan dependency injection yang ringan.
    \item Memiliki navigasi dan route management terintegrasi.
    \item Memiliki HTTP client bawaan.
    \item Mengurangi boilerplate code.
    \item Membantu membuat UI reaktif dengan lebih mudah.
\end{itemize}

\section{Arsitektur Aplikasi}
Aplikasi akan menggunakan arsitektur yang disarankan GetX dengan struktur folder berikut:
\begin{verbatim}
lib/
	| --- main.dart
	|-- app/
	|   `-- app.dart
	|-- core/
	|   |-- constants/
	|   |-- theme/
	|   |-- utils/
	|   `-- values/
	|-- data/
	|   |-- models/
	|   |-- providers/
	|   |   `-- supabase_provider.dart
	|   `-- services/
	|       |-- auth_service.dart
	|       |-- product_service.dart
	|       `-- report_service.dart
	|-- modules/
	|   |-- auth/
	|   |   |-- controllers/
	|   |   |-- views/
	|   |   `-- bindings/
	|   |-- home/
	|   |-- product/
	|   |-- scanner/
	|   `-- report/
	`-- routes/
 	   `-- app_pages.dart
\end{verbatim}

\section{Integrasi dengan Supabase}
Supabase akan digunakan untuk semua fungsionalitas berikut:
\begin{itemize}[leftmargin=1.5cm]
    \item \textbf{Auth:} Menggunakan Supabase Auth untuk manajemen autentikasi.
    \item \textbf{Database:} Penyimpanan data produk, transaksi, dan user di Supabase PostgreSQL.
    \item \textbf{API:} REST API untuk komunikasi antara aplikasi dan database.
    \item \textbf{Realtime:} Menggunakan fitur realtime subscription untuk notifikasi.
\end{itemize}

\subsection{HTTP Request dengan Supabase}
Aplikasi akan menjalankan HTTP request melalui Supabase SDK untuk Flutter untuk:
\begin{itemize}[leftmargin=1.5cm]
    \item Autentikasi (login, register).
    \item CRUD operasi pada tabel (produk, transaksi).
    \item Query data untuk laporan dan statistik.
\end{itemize}

\section{Desain Database (Supabase)}
\subsection{Tabel Users}
\begin{itemize}[leftmargin=1.5cm]
    \item \textbf{id:} UUID, primary key.
    \item \textbf{username:} text, unique.
    \item \textbf{email:} text, unique.
    \item \textbf{created\_at:} timestamp.
\end{itemize}

\subsection{Tabel Products}
\begin{itemize}[leftmargin=1.5cm]
    \item \textbf{id:} UUID, primary key.
    \item \textbf{name:} text.
    \item \textbf{barcode:} text.
    \item \textbf{category:} text.
    \item \textbf{price:} numeric.
    \item \textbf{stock:} integer.
    \item \textbf{min\_stock:} integer.
    \item \textbf{user\_id:} UUID, foreign key.
    \item \textbf{created\_at:} timestamp.
    \item \textbf{updated\_at:} timestamp.
\end{itemize}

\subsection{Tabel Transactions}
\begin{itemize}[leftmargin=1.5cm]
    \item \textbf{id:} UUID, primary key.
    \item \textbf{product\_id:} UUID, foreign key.
    \item \textbf{type:} text --- ``in'' atau ``out''.
    \item \textbf{quantity:} integer.
    \item \textbf{notes:} text.
    \item \textbf{user\_id:} UUID, foreign key.
    \item \textbf{created\_at:} timestamp.
\end{itemize}

\section{Manajemen Risiko}
\begin{itemize}[leftmargin=1.5cm]
    \item Keterlambatan karena single developer.
    \item Ketergantungan pada koneksi internet (Supabase).
    \item Kompatibilitas library kamera pada iOS.
    \item Kurva pembelajaran GetX.
\end{itemize}

\end{document}
